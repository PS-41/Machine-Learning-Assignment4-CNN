\documentclass{article}
\usepackage{amsmath}
\usepackage{hyperref}
\usepackage{geometry}

\geometry{margin=1in}

\title{Feedback on Skeleton Code for Assignment}
\author{Prakhar Suryavansh}
\date{}

\begin{document}

\maketitle

\section*{Feedback on Skeleton Code}

In this assignment, the skeleton code provided a structured and organized starting point, which was beneficial for implementing and fine-tuning the CNN model. Here are the specific aspects I found helpful:

\begin{itemize}
  \item \textbf{Modular Structure:} The skeleton code's modular approach, with distinct functions for tasks such as \texttt{train\_and\_evaluate} and \texttt{tune\_hyperparameters}, helped keep the code organized. This modularity made it easier to understand the purpose of each part.

  \item \textbf{Parameterization for Hyperparameter Tuning:} The skeleton's framework for passing hyperparameters into the \texttt{CNNModel} and \texttt{train\_and\_evaluate} functions streamlined the process of experimenting with different configurations. I used the similar way in my ipynb file by creating the CNNModel class that takes hyperparameters. This parameterization saved time by enabling easy adjustments to layer structure, optimizer choice, and other key settings without extensive code modification.

  \item \textbf{Clarity of Intent:} By providing a clear flow and specific function placeholders, the skeleton code guided me in implementing each part of the assignment without being overly restrictive.

\end{itemize}

Overall, the skeleton code was helpful and contributed positively to completing the assignment. Its clear structure and modular design facilitated a smooth and organized workflow, allowing me to focus on implementing rather than worrying about organization. I believe the skeleton code added value to the project and made the task more manageable and efficient.

\end{document}
